\href{https://drone.io/github.com/shrimpboyho/openrobot/latest}{\tt !\mbox{[}Build Status\mbox{]}(https\-://drone.\-io/github.\-com/shrimpboyho/openrobot/status.\-png)}

{\ttfamily openrobot} is a simple platform for programming N\-X\-T robots built atop of {\ttfamily Ne\-X\-T Btye Codes} (N\-B\-C). At it's core, {\ttfamily openrobot} takes simple source code as input and generates C output. This C output is then fed into the {\ttfamily nbc} compiler where N\-X\-T instructions can be generated.

\subsection*{The Language}

{\ttfamily openrobot} has a language of its own that is heavily based off Python. Here is some sample source code\-:

```python \section*{This is a test program to be parsed}

speed = 4

def \hyperlink{output_8c_ae2b25e760e226c4f8b23bbe4eba84677}{move()}\-: speed = speed + 1 print speed

def \hyperlink{output_8c_a9a2af8e2cd81255d3bf384db4a382807}{main()}\-: \hyperlink{output_8c_ae2b25e760e226c4f8b23bbe4eba84677}{move()} ```

\subsection*{How To Use}

Setting up {\ttfamily openrobot} is quite easy. Make sure you have {\ttfamily Python 2.\-7}.

```bash git clone \href{https://github.com/shrimpboyho/openrobot.git}{\tt https\-://github.\-com/shrimpboyho/openrobot.\-git} cd openrobot ```

From here you can run the tests.

```bash python robotpy \hyperlink{test_8py}{tests/test.\-py} ```

\subsection*{Why?}

{\ttfamily openrobot} was created with education in mind. By exposing students and hobbyists to a simple programming language, the transition from G\-U\-I based programming to text based programming can be easily done. {\ttfamily openrobot} avoids archaic syntax and represents logic in a manner heavily based off of {\ttfamily python}. 